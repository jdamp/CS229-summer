\item \points{10} {\bf PCA} 

In class, we showed that PCA finds the ``variance maximizing'' directions onto
which to project the data.  In this problem, we find another interpretation of PCA. 

Suppose we are given a set of points $\{x^{(1)},\ldots,x^{(\nexp)}\}$. Let us
assume that we have as usual preprocessed the data to have zero-mean and unit variance
in each coordinate.  For a given unit-length vector $u$, let $f_u(x)$ be the 
projection of point $x$ onto the direction given by $u$.  I.e., if 
${\cal V} = \{\alpha u : \alpha \in \Re\}$, then 
\[
f_u(x) = \arg \min_{v\in {\cal V}} ||x-v||^2.
\]
Show that the unit-length vector $u$ that minimizes the 
mean squared error between projected points and original points corresponds
to the first principal component for the data. I.e., show that
$$ \arg \min_{u:u^Tu=1} \sum_{i=1}^\nexp \|x^{(i)}-f_u(x^{(i)})\|_2^2 \ .$$
gives the first principal component.


{\bf Remark.} If we are asked to find a $k$-dimensional subspace onto which to
project the data so as to minimize the sum of squares distance between the
original data and their projections, then we should choose the $k$-dimensional
subspace spanned by the first $k$ principal components of the data.  This problem
shows that this result holds for the case of $k=1$.

\ifnum\solutions=1 {
  \begin{answer}
First, it is easy to see that
\[
f_u(x) = \arg \min_{v\in {\cal V}} ||x-v||^2 = \arg \min_{\alpha\in {\mathbb{R}}} ||x-\alpha u||^2
\]
is solved by $\alpha_{\text{min}}= x^Tu $:
\[
\frac{\mathrm{d}}{\mathrm{d}\alpha} ||x-\alpha u||^2 = -2x^Tu + 2\alpha \underbrace{u^Tu}_{=1} \overset{!}{=}0
\Rightarrow \alpha_{\text{min}} = x^T u
 \]
Using this, the mean squared error between projected and original points can be written as
\begin{align*}
    \arg \min_{u:u^Tu=1} &\sum_{i=1}^\nexp \|x^{(i)}-f_u(x^{(i)})\|_2^2 
    = \arg \min_{u:u^Tu=1} \sum_{i=1}^\nexp \|x^{(i)}-\alpha_{\text{min}}^{(i)}u\|_2^2  \\
    &= \arg \min_{u:u^Tu=1} \sum_{i=1}^\nexp \left((x^{(i)})^2- 2\alpha_{\text{min}}^{(i)} (x^{(i)})^T u + (\alpha_{\text{min}}^{(i)}u)^2\right) \\
    &= \arg \min_{u:u^Tu=1} \sum_{i=1}^\nexp \left(\underbrace{(x^{(i)})^2}_{\text{constant}}- 2(\alpha_{\text{min}}^{(i)})^2 + (\alpha_{\text{min}}^{(i)})^2\underbrace{u^2}_{=1}\right) \\
    &= \arg \min_{u:u^Tu=1} \sum_{i=1}^\nexp \left(- 2(\alpha_{\text{min}}^{(i)})^2 + \alpha_{\text{min}}^2\right) \\
    &= \arg \max_{u:u^Tu=1} \sum_{i=1}^\nexp \alpha_{\text{min}}^2 = \arg \max_{u:u^Tu=1} \sum_{i=1}^\nexp ((x^{(i)})^T u)^2
\end{align*}
which is exactly the definition of the first principal component from the script.
\end{answer}

} \fi

  
