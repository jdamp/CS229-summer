\item \points{40} {\bf Linear Classifiers (logistic regression and GDA)}

In this problem, we cover two probabilistic linear classifiers we have
covered in class so far. First, a discriminative linear classifier: logistic
regression. Second, a generative linear classifier: Gaussian discriminant
analysis (GDA). Both the algorithms find a linear decision boundary that
separates the data into two classes, but make different assumptions. Our goal
in this problem is to get a deeper understanding of the similarities and
differences (and, strengths and weaknesses) of these two algorithms.

For this problem, we will consider two datasets, along with starter codes provided in the following
files:
\begin{center}
\begin{itemize} %[label=\roman*.]
	\item \url{src/linearclass/ds1_{train,valid}.csv}
	\item \url{src/linearclass/ds2_{train,valid}.csv}
        \item \url{src/linearclass/logreg.py}
        \item \url{src/linearclass/gda.py}
\end{itemize}
\end{center}
Each file contains $\nexp$ examples, one example $(x^{(i)}, y^{(i)})$ per row.
In particular, the $i$-th row contains columns $x^{(i)}_0\in\Re$,
$x^{(i)}_1\in\Re$, and $y^{(i)}\in\{0, 1\}$. In the subproblems that follow, we
will investigate using logistic regression and Gaussian discriminant analysis
(GDA) to perform binary classification on these two datasets.

\begin{enumerate}
	\item \subquestionpoints{10}

%\textbf{TBD RETIRE THIS QUESTION? Q4c is a more generic version of this}

In lecture we saw the average empirical loss for logistic regression:
\begin{equation*}
	J(\theta)
	= -\frac{1}{\nexp} \sum_{i=1}^\nexp \left(y^{(i)}\log(h_{\theta}(x^{(i)}))
		+  (1 - y^{(i)})\log(1 - h_{\theta}(x^{(i)}))\right),
\end{equation*}
where $y^{(i)} \in \{0, 1\}$, $h_\theta(x) = g(\theta^T x)$ and
$g(z) = 1 / (1 + e^{-z})$.

Find the Hessian $H$ of this function, and show that for any vector $z$, it
holds true that
%
\begin{equation*}
    z^T H z \ge 0.
\end{equation*}
%
{\bf Hint:} You may want to start by showing that
$\sum_i\sum_j z_i x_i x_j z_j = (x^Tz)^2 \geq 0$. Recall also that
$g'(z) = g(z)(1-g(z))$.

{\bf Remark:} This is one of the standard ways of showing that the matrix $H$
is positive semi-definite, written ``$H \succeq 0$.''  This implies that $J$ is
convex, and has no local minima other than the global one. If you have some
other way of showing $H \succeq 0$, you're also welcome to use your method
instead of the one above.


        \ifnum\solutions=1 {
            \begin{answer}
Remember that
\begin{equation*}
    \frac{\mathrm{d}}{\mathrm{d}z}g(z) = g(z)(1-g(z)),
\end{equation*}
therefore
\begin{equation*}
    \nabla_{\theta} h_\theta(x) =  \nabla_{\theta} \frac{1}{1+\exp\left(-\theta^Tx\right)} = - x\cdot h_\theta(x) (1 - h_\theta(x))
\end{equation*}

First calculate the gradient:
\begin{align*}
    \nabla_{\theta}J(\theta) &= -\frac{1}{\nexp} \sum_{i=1}^\nexp \left(y^{(i)}\log(h_{\theta}(x^{(i)}))
    +  (1 - y^{(i)})\log(1 - h_{\theta}(x^{(i)}))\right) \\
    &= -\frac{1}{\nexp} \sum_{i=1}^\nexp \left(y^{(i)} (-x^{(i)})\cdot (1- h_\theta(x^{(i)}))
    -  (1 - y^{(i)})\cdot(-x^{(i)})\cdot h_\theta(x^{(i)})\right) \\
    &= \frac{1}{\nexp} \sum_{i=1}^\nexp \left(y^{(i)} - h_\theta(x^{(i)})\right)\cdot x^{(i)} \\
\end{align*}
Then the Hessian matrix:
\begin{align*}
    H_{jk}(\theta) &= \frac{\partial^2}{\partial\theta_j\partial\theta_k} J(\theta)
    = \frac{\partial}{\partial\theta_k} \frac{1}{\nexp} \sum_{i=1}^\nexp \left(y^{(i)} - h_\theta(x^{(i)})\right)\cdot x^{(i)}_j \\
    &= \frac{1}{\nexp} \sum_{i=1}^\nexp \left(- h_\theta(x^{(i)}) (1-h_\theta(x^{(i)}))\cdot (-x^{(i)}_k)\right)\cdot x^{(i)}_j \\
    &= \frac{1}{\nexp} \sum_{i=1}^\nexp \left(h_\theta(x^{(i)}) (1-h_\theta(x^{(i)}))\right)\cdot x^{(i)}_jx^{(i)}_k \\
    \Rightarrow H &= \sum_{i=1}^\nexp h_\theta(x^{(i)}) (1-h_\theta(x^{(i)})) \cdot x^{(i)} (x^{(i)})^T
\end{align*}
Now prove $z^THz\geq 0$ for all $z$. For this, use Einstein sum convention (implicit summation over double subscript indices)

\begin{align*}
    z^THz &= z_j H_{jk} z_k = z_j \sum_{i=1}^\nexp h_\theta(x^{(i)}) (1-h_\theta(x^{(i)})) \cdot x^{(i)}_j x^{(i)}_k z_k \\
    &= \sum_{i=1}^\nexp h_\theta(x^{(i)}) (1-h_\theta(x^{(i)})) z_jx^{(i)}_j x^{(i)}_k z_k \\
    &= \sum_{i=1}^\nexp h_\theta(x^{(i)}) (1-h_\theta(x^{(i)})) (z^T\cdot x^{(i)}) ((x^{(i)})^T z) \\
    &= \sum_{i=1}^\nexp \underbrace{h_\theta(x^{(i)}) (1-h_\theta(x^{(i)}))}_{\geq 0} \underbrace{(z^T x^{(i)})^2}_{\geq 0}  \\
    &\Rightarrow H \succeq 0 \\
\end{align*}
\end{answer}

        } \fi

	\item \subquestionpoints{5} \textbf{Coding problem.}
Follow the instructions in \texttt{src/linreg/logreg.py} to train a
logistic regression classifier using Newton's Method.
Starting with $\theta = \vec{0}$, run Newton's Method until the updates to
$\theta$ are small: Specifically,  train until the first iteration $k$ such
that $\|\theta_{k} - \theta_{k-1}\|_1 < \epsilon$, where
$\epsilon = 1\times 10^{-5}$. Make sure to write your model's predicted probabilities on
the validation set to the file specified in the code.

Include a plot of the \textbf{validation data} with $x_1$ on the horizontal axis and $x_2$ on the vertical axis.
To visualize the two classes, use a different symbol for examples $x^{(i)}$
with $y^{(i)} = 0$ than for those with $y^{(i)} = 1$. On the same figure, plot the decision boundary
found by logistic regression (i.e, line corresponding to $p(y|x) = 0.5$).


        \ifnum\solutions=1 {
            \begin{answer}
    \begin{figure}[h]
        \centering
        \includegraphics*[width=.7\linewidth]{../src/linearclass/logreg_pred_1.pdf}
        \caption{Decision boundary for logistic regression for the first dataset}        
    \end{figure}
\end{answer}

        } \fi


	\item \subquestionpoints{5}
Recall that in GDA we model the joint distribution of $(x, y)$ by the following
equations:
%
\begin{eqnarray*}
	p(y) &=& \begin{cases}
	\phi & \mbox{if~} y = 1 \\
	1 - \phi & \mbox{if~} y = 0 \end{cases} \\
	p(x | y=0) &=& \frac{1}{(2\pi)^{\di/2} |\Sigma|^{1/2}}
		\exp\left(-\frac{1}{2}(x-\mu_{0})^T \Sigma^{-1} (x-\mu_{0})\right) \\
	p(x | y=1) &=& \frac{1}{(2\pi)^{\di/2} |\Sigma|^{1/2}}
		\exp\left(-\frac{1}{2}(x-\mu_1)^T \Sigma^{-1} (x-\mu_1) \right),
\end{eqnarray*}
%
where $\phi$, $\mu_0$, $\mu_1$, and $\Sigma$ are the parameters of our model.

Suppose we have already fit $\phi$, $\mu_0$, $\mu_1$, and $\Sigma$, and now
want to predict $y$ given a new point $x$. To show that GDA results in a
classifier that has a linear decision boundary, show the posterior distribution
can be written as
%
\begin{equation*}
	p(y = 1\mid x; \phi, \mu_0, \mu_1, \Sigma)
	= \frac{1}{1 + \exp(-(\theta^T x + \theta_0))},
\end{equation*}
%
where $\theta\in\Re^\di$ and $\theta_{0}\in\Re$ are appropriate functions of
$\phi$, $\Sigma$, $\mu_0$, and $\mu_1$.


        \ifnum\solutions=1 {
            \begin{answer}    
Using Bayes we can write:
\begin{align*}
&p(y=1 \mid x;\phi,\mu_0,\mu_1,\Sigma) = \frac{p(x\mid y=1; \mu_1, \Sigma)\cdot p(y=1;\phi) } {p(x;\phi,\mu_0,\mu_1,\Sigma)} \\
&= \frac{p(x\mid y=1; \mu_1, \Sigma)\cdot p(y;\phi) } {p(x \mid y=0;\phi,\mu_0,\mu_1,\Sigma)\cdot p(y=0;\phi) + p(x \mid y=1;\phi,\mu_0,\mu_1,\Sigma)\cdot p(y=1;\phi)} \\
&= \frac{1}{1 + \frac{p(y=0;\phi)}{p(y=1;\phi)}\cdot \frac{p(x \mid y=0;\phi,\mu_0,\mu_1,\Sigma)}{p(x \mid y=1;\phi,\mu_0,\mu_1,\Sigma)}}   
\end{align*}
Inserting the definitions:
\begin{align*}
&p(y=1 \mid x;\phi,\mu_0,\mu_1,\Sigma) = \frac{1}{1+ \frac{1-\phi}{\phi} \cdot \exp\left(-\frac{1}{2}(x-\mu_{0})^T \Sigma^{-1} (x-\mu_{0}) + \frac{1}{2}(x-\mu_{1})^T \Sigma^{-1} (x-\mu_{1})\right)} \\
&= \frac{1}{1+ \exp\left(\log\left(\frac{1-\phi}{\phi}\right)-\frac{1}{2}(x-\mu_{0})^T \Sigma^{-1} (x-\mu_{0})+\frac{1}{2}(x-\mu_{1})^T \Sigma^{-1} (x-\mu_{1})\right)} \\
\end{align*}
Expand the second and third term in the denominator:
\begin{align*}
&-(x-\mu_{0})^T \Sigma^{-1} (x-\mu_{0})+(x-\mu_{1})^T \Sigma^{-1} (x-\mu_{1})  = \\
& -x^T\Sigma^{-1} x + x^T\Sigma^{-1}\mu_0 + \mu_0^T\Sigma^{-1} x - \mu_0^T\Sigma^{-1}\mu_0 + x^T\Sigma^{-1} x - x^T\Sigma^{-1}\mu_1 - \mu_1^T\Sigma^{-1} x + \mu_1^T\Sigma^{-1}\mu_1 \\
&= 2 (\mu_0^T\Sigma^{-1} - \mu_1^T\Sigma^{-1})x - \mu_0^T\Sigma^{-1}\mu_0  + - \mu_1^T\Sigma^{-1}\mu_1 \\
\end{align*}
Therefore, by chosing
\begin{align*}
    \theta &= -\Sigma^{-1} (\mu_0 - \mu_1) \\
    \theta_0 &= - \log\left(\frac{1-\phi}{\phi}\right) + \frac{1}{2}\left(\mu_0^T\Sigma^{-1}\mu_0   - \mu_1^T\Sigma^{-1}\mu_1\right)
\end{align*}
one can achieve the desired form of the posterior distribution.
\end{answer}

        }\fi

	\item \subquestionpoints{7} Given the dataset, we claim that the maximum
  likelihood estimates of the parameters are given by
  \begin{eqnarray*}
    \phi &=& \frac{1}{\nexp} \sum_{i=1}^\nexp 1\{y^{(i)} = 1\} \\
\mu_{0} &=& \frac{\sum_{i=1}^\nexp 1\{y^{(i)} = {0}\} x^{(i)}}{\sum_{i=1}^\nexp
1\{y^{(i)} = {0}\}} \\
\mu_1 &=& \frac{\sum_{i=1}^\nexp 1\{y^{(i)} = 1\} x^{(i)}}{\sum_{i=1}^\nexp 1\{y^{(i)}
= 1\}} \\
\Sigma &=& \frac{1}{\nexp} \sum_{i=1}^\nexp (x^{(i)} - \mu_{y^{(i)}}) (x^{(i)} -
\mu_{y^{(i)}})^T
  \end{eqnarray*}
  The log-likelihood of the data is
  \begin{eqnarray*}
\ell(\phi, \mu_{0}, \mu_1, \Sigma) &=& \log \prod_{i=1}^\nexp p(x^{(i)} , y^{(i)};
\phi, \mu_{0}, \mu_1, \Sigma) \\
&=& \log \prod_{i=1}^\nexp p(x^{(i)} | y^{(i)}; \mu_{0}, \mu_1, \Sigma) p(y^{(i)};
\phi).
  \end{eqnarray*}
By maximizing $\ell$ with respect to the four parameters,
prove that the maximum likelihood estimates of $\phi$, $\mu_{0}, \mu_1$, and
$\Sigma$ are indeed as given in the formulas above.  (You may assume that there
is at least one positive and one negative example, so that the denominators in
the definitions of $\mu_{0}$ and $\mu_1$ above are non-zero.)


        \ifnum\solutions=1 {
            \begin{answer}
Let \(\mathcal{X}^1 = \{i=1,\ldots,n \mid y_i = 1\} \) and \(\mathcal{X}^0 = \{i=1,\ldots,n \mid y_i = 0\} \) and $n^1$, $n^0$ denote the number of elements in these sets.
Then:
\begin{align*}
\ell(\phi, \mu_{0}, \mu_1, \Sigma) &= \sum_{i\in\mathcal{X}^1} p(x^{(i)} | 1; \mu_{0}, \mu_1, \Sigma) p(1;\phi) + \sum_{i\in\mathcal{X}^0} p(x^{(i)} | 0; \mu_{0}, \mu_1, \Sigma) p(0;\phi)    \\
&= \frac{n}{2}\log\left((2\pi)^{d}\det\left({\Sigma^{-1}}\right)\right) + \biggl[ \sum_{i\in\mathcal{X}^1} \frac{1}{2}(x^{(i)}-\mu_{1})^T \Sigma^{-1} (x^{(i)}-\mu_{1}) + \log\phi + \\
& \sum_{i\in\mathcal{X}^0} \frac{1}{2}(x^{(i)}-\mu_{0})^T \Sigma^{-1} (x^{(i)}-\mu_{0}) + \log\left(1-\phi\right)   \biggr]
\end{align*}
Find the optimal parameters by maximizing the log-likelihood $\ell$.
First for $\phi$, neglecting factors independent on $\phi$:
\begin{align*}
    &\frac{\partial\ell}{\partial\phi} \propto \sum_{i\in\mathcal{X}^1}\frac{1}{\phi} - \sum_{i\in\mathcal{X}^0}\frac{1}{1-\phi} = n^1 \frac{1}{\phi} - n^0 \frac{1}{1-\phi} \overset{!}{=} 0 \\
    &\Rightarrow \hat{\phi} = \frac{n^1}{n} = \frac{1}{\nexp} \sum_{i=1}^\nexp 1\{y^{(i)} = 1\}
\end{align*}
Then for $\mu^0$ (and analogously for $\mu_1$), making use of $\nabla_x x^T\Sigma x = 2\Sigma x$:
\begin{align*}
    &\nabla_{\mu_0}\ell = \frac{1}{2}\sum_{i\in\mathcal{X}^0} \nabla_{\mu_0}\left((x^{(i)}-\mu_{0})^T \Sigma^{-1} (x^{(i)}-\mu_{0})\right) = \sum_{i\in\mathcal{X}^0} \Sigma^{-1}(x^{(i)}-\mu_0) \overset{!}{=} 0 \\
    &\Rightarrow \hat{\mu_0} = \frac{1}{n^0} \sum_{i\in\mathcal{X}^0} x^{(i)} = \frac{\sum_{i=1}^\nexp 1\{y^{(i)} = {0}\} x^{(i)}}{\sum_{i=1}^\nexp
    1\{y^{(i)} = {0}\}} \\
    &\Rightarrow \hat{\mu_1} = \frac{1}{n^1} \sum_{i\in\mathcal{X}^1} x^{(i)} = \frac{\sum_{i=1}^\nexp 1\{y^{(i)} = {1}\} x^{(i)}}{\sum_{i=1}^\nexp
    1\{y^{(i)} = {1}\}}
\end{align*}
Last for $\Sigma$ we need to make use of the following identities from the script for a (symmetric) matrix $A$:
\begin{align*}
    \nabla_A \det A = \det A\cdot A^{-T} \\
    \nabla_A \log \det A = A^{-T} \\
    \Rightarrow \nabla_A\log\det A^{-1} = - A^{-T} = -A^{-1}
\end{align*}
In addition, we need to know $\nabla_A A^{-1}$:
\begin{align*}
    0 = \nabla_A \mathbb{I} = \nabla_A (A\cdot A^{-1}) =  A^{-1} + A \nabla_A A^{-1} \Rightarrow \nabla_A A^{-1} = - A^{-1} A^{-1}
\end{align*}
And finally, using the fact that the quadratic form is a scalar, and hence equivalent to its trace:
\begin{align*}
    \nabla_A z^T A^{-1} z = \nabla_A\mathrm{tr}(z^T A^{-1} z) = \nabla_A\mathrm{tr}(A^{-1} z\cdot z^T) =  - A^{-1} A^{-1} z\cdot z^T
\end{align*}

Then
\begin{align*}
    \nabla_\Sigma\ell = -\frac{n}{2}\Sigma^{-1} - \biggl[ \sum_{i\in\mathcal{X}^1} \Sigma^{-1}\Sigma^{-1} \frac{1}{2}(x^{(i)}-\mu_{1}) (x^{(i)}-\mu_{1})^T
    & \sum_{i\in\mathcal{X}^0} \frac{1}{2}\Sigma^{-1}\Sigma^{-1} (x^{(i)}-\mu_{0}) (x^{(i)}-\mu_{0})^T  \biggr]
\end{align*}
Setting this equal to zero and multiplying twice from the left with $\Sigma$ yields:
\begin{align*}
    \hat{\Sigma} &= \frac{1}{n}\sum_{i\in\mathcal{X}^1} (x^{(i)}-\mu_{1}) (x^{(i)}-\mu_{1})^T
     \sum_{i\in\mathcal{X}^0} (x^{(i)}-\mu_{0}) (x^{(i)}-\mu_{0})^T \\
     &= \frac{1}{\nexp} \sum_{i=1}^\nexp (x^{(i)} - \mu_{y^{(i)}}) (x^{(i)} -
     \mu_{y^{(i)}})^T
\end{align*}
\hfill\ensuremath{\square}
\end{answer}

        } \fi

	\item \subquestionpoints{5} \textbf{Coding problem.}
In \texttt{src/linreg/gda.py}, fill in the code to
calculate $\phi$, $\mu_{0}$, $\mu_{1}$, and $\Sigma$, use these parameters
to derive $\theta$, and use the resulting GDA model to make predictions on the
validation set. Make sure to write your model's predictions on
the validation set to the file specified in the code.

Include a plot of the \textbf{validation data} with $x_1$ on the horizontal axis and $x_2$ on the vertical axis.
To visualize the two classes, use a different symbol for examples $x^{(i)}$
with $y^{(i)} = 0$ than for those with $y^{(i)} = 1$. On the same figure, plot the decision boundary
found by GDA (i.e, line corresponding to $p(y|x) = 0.5$).


        \ifnum\solutions=1 {
            \FloatBarrier

\begin{answer}
    \begin{figure}[h]
        \centering
        \includegraphics*[width=.7\linewidth]{../src/linearclass/gda_pred_1.pdf}
        \caption{Decision boundary for GDA for the first dataset}        
    \end{figure}

\end{answer}

\FloatBarrier
        } \fi

	\item \subquestionpoints{2}
For Dataset 1, compare the validation set plots obtained in part (b) and part (e)
from logistic regression and GDA respectively, and briefly comment on your observation
in a couple of lines.


        \ifnum\solutions=1 {
            \begin{answer}
    Logistic regression performs better than GDA on this dataset, with a higher accuracy and a more reasonable looking classification boundary.
    
\end{answer}

        } \fi

	\item \subquestionpoints{5}
Repeat the steps in part (b) and part (e) for Dataset 2. Create similar plots on
the \textbf{validation set} of Dataset 2 and include those plots in your writeup.

On which dataset does GDA seem to
perform worse than logistic regression? Why might this be the case?


        \ifnum\solutions=1{
            \FloatBarrier
\begin{answer}
    \begin{figure}[h]
        \centering
        \includegraphics*[width=.7\linewidth]{../src/linearclass/logreg_pred_2.pdf}
        \caption{Decision boundary for logistic regression for the second dataset}        
    \end{figure}

    \begin{figure}[h]
        \centering
        \includegraphics*[width=.7\linewidth]{../src/linearclass/gda_pred_2.pdf}
        \caption{Decision boundary for GDA for the second dataset}        
    \end{figure}

    GDA performs worse on the first data set, as the data values $x^{(i)}$ are not Gaussian distributed for this data.
The second data set looks more Gaussian, leading to an improved performance of the GDA classification.
\end{answer}

\FloatBarrier

        }\fi

	\item \points{1} For the dataset where GDA performed worse in
parts (f) and (g), can you find a transformation of the $x^{(i)}$'s such
that GDA performs significantly better? What might this transformation be?


        \ifnum\solutions=1{
            \begin{answer}
    The first data set seems to roughly behave as if $x_2$ behaves exponentially compared to the second data set.
    A possible transformation could therefore be $x_2\rightarrow \log x_2$.
\end{answer}

        }\fi

\end{enumerate}
