\begin{answer}
Let $\mathcal{X}^+=\{x \mid p((t^{(i)})=1 \mid x^{(i)} = x) = 1 \}$ denote the space of all positive examples and $\mathcal{X}^-=\{x \mid p((t^{(i)})=0 \mid x^{(i)} = x) = 1 \}$.
By the assumption $p(t^{(i)}=1| x^{i})\in\{0,1\}$, we can follow that $\mathcal{X}^-$ is the complement of $\mathcal{X^+}$.
It follows that $p(x^{(i)}\in \mathcal{X}^- \mid y^{(i)}=1)= 0$, due to the third and fourth equation of our initial assumptions and the.
Consequently
\begin{equation}
    p(x^{(i)}\in \mathcal{X}^+ \mid y^{(i)}=1)= 1.
\end{equation}
Now using the result from 2(d) implies for $x\in\mathcal{X}^+$:
\begin{align*}
    h(x) &= \alpha p(t^{(i)}=1| x^{(i)}=x) = \alpha  \\
    &\Rightarrow p(h(x^{(i)})=\alpha \mid x^{(i)}=x) = 1
\end{align*}
By definition of conditional probability, this implies
\begin{equation*}
    p(h(x^{(i)})=\alpha \mid x^{(i)}=x, y^{(i)}=1) = 1
\end{equation*}
Putting this together:
\begin{align*} 
    p(h(x^{(i)}) =\alpha \mid y^{(i)}=1) &= p(h(x^{(i)})=\alpha \mid y^{(i)}=1, x^{(i)}\in \mathcal{X}^+)\cdot p(x^{(i)}\in \mathcal{X}^+ \mid y^{(i)}=1) \\
                 &= p(x^{(i)}\in \mathcal{X}^+ \mid y^{(i)}=1) \\
                 &= 1 \cdot 1 = 1
\end{align*}
Therefore $\alpha = \mathbb{E}[h(x^{(i)})\mid y^{(i)}=1]$.
\end{answer}
