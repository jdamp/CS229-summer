\item \points{15} {\bf Convexity of Generalized Linear Models}

In this question we will explore and show some nice properties of Generalized
Linear Models, specifically those related to its use of Exponential Family
distributions to model the output.

Most commonly, GLMs are trained by using the negative log-likelihood (NLL) as
the loss function. This is mathematically equivalent to Maximum Likelihood
Estimation (\emph{i.e.,} maximizing the log-likelihood is equivalent to
minimizing the negative log-likelihood). In this problem, our goal is to show
that the NLL loss of a GLM is a \textit{convex} function w.r.t the model parameters. As
a reminder, this is convenient because a convex function is one for which any
local minimum is also a global minimum, and there are extensive research on how to optimize various types of convex functions efficiently with various algorithms such as stochastic gradient descent. 

To recap, an exponential family distribution is one whose probability density
can be represented
%
\begin{equation*}
    p(y; \eta) = b(y)\exp(\eta^TT(y) - a(\eta)),
\end{equation*}
%
where $\eta$ is the \emph{natural parameter} of the distribution. Moreover, in
a Generalized Linear Model, $\eta$ is modeled as $\theta^Tx$, where $x \in
\mathbb{R}^\di$ are the input features of the example, and $\theta \in
\mathbb{R}^\di$ are learnable parameters. In order to show that the NLL loss is
convex for GLMs, we break down the process into sub-parts, and approach them
one at a time. Our approach is to show that the second derivative (\emph{i.e.,}
Hessian) of the loss w.r.t the model parameters is Positive Semi-Definite (PSD)
at all values of the model parameters. We will also show some nice properties
of Exponential Family distributions as intermediate steps.

For the sake of convenience we restrict ourselves to the case where $\eta$ is
a scalar. Assume $p(Y|X;\theta )\sim \text{ExponentialFamily}(\eta)$, where
$\eta \in\mathbb{R}$ is a scalar, and $T(y) = y$. This makes the exponential
family representation take the form
%
\begin{equation*}
    p(y ; \eta) = b(y)\exp(\eta y - a(\eta)).
\end{equation*}
%
\begin{enumerate}
    \item \subquestionpoints{5}
Derive an expression for the mean of the distribution. Show that
$\mathbb{E}[Y; \eta] = \frac{\partial}{\partial\eta}a(\eta)$ (note that
$\mathbb{E}[Y; \eta] = \mathbb{E}[Y | X; \theta]$ since $\eta = \theta^T x$).
In other words, show that the mean of an exponential family distribution is the
first derivative of the log-partition function with respect to the natural
parameter.

\textbf{Hint:} Start with observing that $\frac{\partial}{\partial \eta} \int
p(y;\eta) dy = \int \frac{\partial}{\partial \eta} p(y;\eta) dy$.


\ifnum\solutions=1{
  \begin{answer}
\begin{align*}
    \int \frac{\partial}{\partial \eta} p(y;\eta) \,dy 
    &= \int \frac{\partial}{\partial \eta} p(y;\eta) \,dy
    = \int \frac{\partial}{\partial \eta} b(y)\exp(\eta y - a(\eta)) \,dy \\
    &= \int \underbrace{b(y) \exp(\eta^TT(y) - a(\eta))}_{p(y;\eta)} \left(y - \frac{\partial a(\eta)}{\partial\eta}\right) \,dy \\
    &=\int p(y;\eta) \left(y - \frac{\partial a(\eta)}{\partial\eta}\right) \,dy 
\end{align*}
Since $p$ is an propability density function, the value of $\int p(y;\eta) \,dy$ must be equal to 1, independent of $\eta$.
Therefore:
\begin{align*}
    \frac{\partial}{\partial \eta} \int p(y;\eta) \,dy  = 0.
\end{align*}
Using the equality given in the hint:
\begin{align*}
    &\frac{\partial}{\partial \eta} \int p(y;\eta) \,dy  =  \int\frac{\partial}{\partial \eta} p(y;\eta) \,dy 
    \Leftrightarrow 0 = \int p(y;\eta) \left(y - \frac{\partial a(\eta)}{\partial\eta}\right) \,dy  \\
    &\Rightarrow \int p(y;\eta) y \,dy = \int p(y;\eta) \frac{\partial a(\eta)}{\partial\eta} \,dy \\
    &\Rightarrow \mathbb{E}\left[Y|\eta\right] = \frac{\partial a(\eta)}{\partial\eta} 
\end{align*}
\end{answer}

}\fi

    \item \subquestionpoints{5}
Next, derive an expression for the variance of the distribution. In particular,
show that $\text{Var}(Y; \eta) = \frac{\partial^2}{\partial\eta^2}a(\eta)$
(again, note that $\text{Var}(Y; \eta) = \text{Var}(Y | X; \theta)$). In
other words, show that the variance of an exponential family distribution is
the second derivative of the log-partition function w.r.t. the natural
parameter.

\textbf{Hint:} Building upon the result in the previous sub-problem can
simplify the derivation.


\ifnum\solutions=1{
  \begin{answer}
Remember that
\begin{equation}
    \label{eq:var}
    \mathrm{Var}(Y;\eta) = \mathbb{E}[Y^2;\eta] - (\mathbb{E}[Y;\eta])^2
\end{equation}

Use the same approach as in part (a):
\begin{align*} 0 &= \int \frac{\partial^2}{\partial \eta^2} p(y;\eta) \,dy  = \int \frac{\partial}{\partial \eta} \left(y-\frac{\partial a(\eta)}{\partial\eta}\right)p(y;\eta) \,dy \\
    &= \int \left( -\frac{\partial^2 a(\eta)}{\partial \eta^2} + \left(y-\frac{\partial a(\eta)}{\partial\eta}\right)^2 \right) p(y;\eta) \,dy \\
    &= \int \left( -\frac{\partial^2 a(\eta)}{\partial \eta^2} + y^2 - 2 y \frac{\partial a(\eta)}{\partial\eta} + \left(\frac{\partial a(\eta)}{\partial\eta}\right)^2 \right) p(y;\eta) \,dy \\
    &= -\frac{\partial^2 a(\eta)}{\partial \eta^2} +  \mathbb{E}[Y^2;\eta] - 2\frac{\partial a(\eta)}{\partial\eta} \mathbb{E}[Y;\eta] + \left(\frac{\partial a(\eta)}{\partial\eta}\right)^2 \\
    &= -\frac{\partial^2 a(\eta)}{\partial \eta^2} +  \mathbb{E}[Y^2;\eta] - 2 (\mathbb{E}[Y;\eta])^2 + \left(\mathbb{E}[Y;\eta]\right)^2 \\
\end{align*}
Combining this result with equation \ref{eq:var} yields the desired result:
\begin{equation*}
    \mathrm{Var}(Y;\eta) = \mathbb{E}[Y^2;\eta] - (\mathbb{E}[Y;\eta])^2 = \frac{\partial^2 a(\eta)}{\partial \eta^2}.
\end{equation*}
\end{answer}

}\fi


    \item \subquestionpoints{5}
Finally, write out the loss function $\ell(\theta)$, the NLL of the
distribution, as a function of $\theta$. Then, calculate the Hessian of the
loss w.r.t $\theta$, and show that it is always PSD. This concludes the proof
that NLL loss of GLM is convex.

\textbf{Hint 1:} Use the chain rule of calculus along with the results of
the previous parts to simplify your derivations.

\textbf{Hint 2:} Recall that variance of any probability distribution is
non-negative.


\ifnum\solutions=1{
	\begin{answer}
Start by writing down the loss function for a single example:
\begin{align*}
    \ell(\theta) = - \log(p(y;\eta)) = - \log(b(y)) - \eta y + a(\eta),
\end{align*}
where $\eta=\theta^T x$.
Use the chain rule to transform the derivative
\begin{equation*}
    \frac{\partial}{\partial\theta_i} = \frac{\partial\eta}{\partial\theta_i} \frac{\partial}{\partial\eta} = x_i \frac{\partial}{\partial\eta}
\end{equation*}
and calculate the second order partial derivatives:
\begin{align*}
    \frac{\partial^2 \ell(\theta)}{\partial\theta_i\partial\theta_j} =  \frac{\partial}{\partial\theta_j} \left(x_i \frac{\partial}{\partial\eta} \ell(\theta) \right)  =
    \frac{\partial}{\partial\theta_j} \left(x_i \left[y + \frac{\partial a(\eta)}{\partial\eta}\right] \right)  = x_i x_j \frac{\partial^2 a(\eta)}{\partial\eta^2}
\end{align*}
Therefore, the Hessian can be written as 
\begin{equation*}
    H_\ell(\theta) = xx^T\frac{\partial^2 a(\eta)}{\partial\eta^2}.
\end{equation*}
To check that $H_\ell$ is PSD, the condition $z^T H_\ell z \geq 0$ has to be checked for all $z$:
\begin{align*}
z^T H_\ell(\theta) z &= \frac{\partial^2 a(\eta)}{\partial\eta^2} z^T (x^T x) z = \frac{\partial^2 a(\eta)}{\partial\eta^2} \sum_{i,j} z_i (x^T x)_{ij} z_j
 = \frac{\partial^2 a(\eta)}{\partial\eta^2} \sum_{i,j} z_i x_i x_j z_j \\
 &= \underbrace{\frac{\partial^2 a(\eta)}{\partial\eta^2}}_{\text{Variance} \geq 0} (z^T x)^2 \geq 0
\end{align*}
Therefore, any GLM model is convex in its model parameters.
\end{answer}

}\fi


\end{enumerate}

\textbf{Remark:} The main takeaways from this problem are:
\begin{itemize}
  \item Any GLM model is convex in its model parameters.
  \item The exponential family of probability distributions are mathematically
  nice. Whereas calculating mean and variance of distributions in general
  involves integrals (hard), surprisingly we can calculate them using
  derivatives (easy) for exponential family.
\end{itemize}
