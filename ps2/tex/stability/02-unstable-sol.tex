\begin{answer}
First plot the features of both datasets:
\begin{figure}[h]
    \begin{minipage}[t]{.49\linewidth}
        \centering
        \includegraphics*[width=\linewidth]{../src/stability/ds1_a.pdf}    
    \end{minipage}
    \hfill
    \begin{minipage}[t]{.49\linewidth}
        \centering
        \includegraphics*[width=\linewidth]{../src/stability/ds1_b.pdf}    
    \end{minipage}
    \caption{Scatterplot of the features of the Datasets A and B.}
\end{figure}
One can see that while dataset A is perfectly linearly seperable, this is not the case for dataset B.

\begin{figure}[h]
    \begin{minipage}[t]{.49\linewidth}
        \centering
        \includegraphics*[width=\linewidth]{../src/stability/gradients_a.pdf}    
    \end{minipage}
    \hfill        
    \begin{minipage}[t]{.49\linewidth}
        \centering
        \includegraphics*[width=\linewidth]{../src/stability/gradients_b.pdf}    
    \end{minipage}
    \caption{Magnitude of the gradient $||\nabla\theta||$ for both datasets as a function of the number of iterations.}
\end{figure}
For dataset A, the gradients vanish quickly over many orders of magnitude, leading to the convergence criterion after about 22000 iterations.
For dataset B however, the gradients stay in the same order of magnitude, therefore the convergence criterion is reached only much, much later.

\end{answer}
