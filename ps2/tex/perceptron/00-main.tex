\item \points{15} {\bf Kernelizing the Perceptron}

Let there be a binary classification problem with $y \in \{0, 1\}$.  The
perceptron uses hypotheses of the form $h_\theta(x) = g(\theta^T x)$, where
$g(z) = \text{sign}(z) = 1$ if $z \ge 0$, $0$ otherwise.  In this problem we
will consider a stochastic gradient descent-like implementation of the
perceptron algorithm where each update to the parameters $\theta$ is made using
only one training example.  However, unlike stochastic gradient descent, the
perceptron algorithm will only make one pass through the entire training set.
The update rule for this version of the perceptron algorithm is given by
\begin{equation*}
  \theta^{(i+1)} :=
	  \theta^{(i)} + \alpha (y^{(i+1)} - h_{\theta^{(i)}}(x^{(i+1)})) x^{(i+1)}
\end{equation*}
where $\theta^{(i)}$ is the value of the parameters after the algorithm has
seen the first $i$ training examples. Prior to seeing any training examples,
$\theta^{(0)}$ is initialized to $\vec{0}$.

\begin{enumerate}
  \item \subquestionpoints{3} Let $K$ be a Mercer kernel corresponding to some
very high-dimensional feature
mapping $\phi$. Suppose $\phi$ is so high-dimensional (say,
$\infty$-dimensional) that it's infeasible to ever represent $\phi(x)$
explicitly.  Describe how you would apply the ``kernel trick'' to the
perceptron to make it work in the high-dimensional feature space $\phi$, but
without ever explicitly computing $\phi(x)$.

[\textbf{Note:} You don't have to worry about the intercept term.  If you like,
think of $\phi$ as having the property that $\phi_0(x) = 1$ so that this is
taken care of.] Your description should specify:
\begin{enumerate}[label=\roman*.]
  \item \subquestionpoints{1} How you will (implicitly) represent the
  high-dimensional
    parameter vector $\theta^{(i)}$, including how the initial value
    $\theta^{(0)} = 0$ is represented (note that $\theta^{(i)}$ is
    now a vector whose dimension is the same as the feature vectors
    $\phi(x)$);
  \item \subquestionpoints{1} How you will efficiently make a prediction on a
  new input
    $x^{(i+1)}$.  I.e., how you will compute
    $h_{\theta^{(i)}}(x^{(i+1)}) = g({\theta^{(i)}}^T \phi(x^{(i+1)}))$,
    using your representation of $\theta^{(i)}$; and
  \item \subquestionpoints{1} How you will modify the update rule given above
  to perform an
  update to $\theta$ on a new training example $(x^{(i+1)}, y^{(i+1)})$;
  \emph{i.e.,} using the update rule corresponding to the feature mapping
  $\phi$:
  \begin{equation*}
  \theta^{(i+1)} :=
	  \theta^{(i)} + \alpha (y^{(i+1)} - h_{\theta^{(i)}}(x^{(i+1)})) \phi(x^{(i+1)})
  \end{equation*}
\end{enumerate}


\ifnum\solutions=1 {
  \begin{answer}
The vector $\theta^{(i)}$ will be represented as a linear combination of the feature vectors $\phi(x^{(j)})$, i.e.
\begin{equation*}
    \theta^{(i)} = \sum_{j=1}^{n} \beta_j^{(i)}\phi(x^{(j)}),
\end{equation*}
where all information is stored in the coefficients $\beta_j^{(i)}$.
For $i=0$, $\beta_j^{(0)}=0\,\forall j$.
For $i>0$, the update rule for $\beta_j^{(i)}$ can be written as a function of the Mercer Kernel $K$, so there is no need to calculate
the feature vectors $\phi$ explicitly:
\begin{equation*}
    \beta_j^{(i+1)} = \beta_j^{(i)} + \alpha \left( y^{(i)} - g\left(\sum_{k=1}^i \beta_k^{(i)}\cdot K\left(x^{(i)}, x^{(k)}\right)\right)\right)
\end{equation*}
Prediction can be made using
\begin{align*}
    h_{\theta^{(i)}}(x^{(i+1)}) = g(\theta^{(i)^T} \phi(x^{i+1}))
     = g \left(\sum_{j=1}^{n} \beta_j^{(i)}{\phi(x^{(j)})^T} \phi(x^{(i+1)}) \right) \\
     = g \left(\sum_{j=1}^{n} \beta_j^{(i)} K\left( x^{(j)}, x^{(i+1)}\right) \right)
\end{align*}
\end{answer}

} \fi

  \item \subquestionpoints{10} Implement your approach by completing the
\texttt{initial\_state}, \texttt{predict}, and \texttt{update\_state} methods
of \texttt{src/perceptron/perceptron.py}.


We provide two kernels, a dot-product kernel and a
radial basis function (RBF) kernel. Run \texttt{src/perceptron/perceptron.py} to train
kernelized perceptrons on \texttt{src/perceptron/train.csv}. The code will then test
the perceptron on \texttt{src/perceptron/test.csv} and save the resulting
predictions in the \texttt{src/perceptron/} folder. Plots will also be saved in
\texttt{src/perceptron/}.


Include the two plots (corresponding to each of the kernels) in your writeup,
and indicate which plot belongs to which kernel.


\ifnum\solutions=1 {
  \begin{answer}
\begin{figure}[h]
    \centering
    \includegraphics[width=0.75\linewidth]{../src/perceptron/perceptron_dot_output.png}
    \caption{Kernelized perceptron classification regions for the dot product kernel}
\end{figure}

\begin{figure}[h]
    \centering
    \includegraphics[width=0.75\linewidth]{../src/perceptron/perceptron_rbf_output.png}
    \caption{Kernelized perceptron classification regions for the RBF kernel}
\end{figure}
\end{answer}

\FloatBarrier
} \fi


  \item \subquestionpoints{2} 

One of the provided kernels performs
extremely poorly in classifying the points. Which kernel performs badly and why
does it fail?



\ifnum\solutions=1 {
  \begin{answer}
    The dot product kernel performs poorly, as the data is not linearly seperable
     and it does not represent the necessary non-linear feature map.
\end{answer}

} \fi

\end{enumerate}
